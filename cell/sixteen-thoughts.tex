%%%%%%%%%%%%%%%%%%%%%%%%%%%%%%%%%%%%%%%%%%%%%%%%%%%%%%%%%%%%%%%%%%%%%%%%
%
% Sixteen Thoughts on Rule 110
%
% Or possibly just programming, melody, lilypond, and TeX practice.

\documentclass[a4paper,10pt,twoside,onecolumn,openany,final]{memoir}
\settrimmedsize{297mm}{210mm}{*}
\setlength{\trimtop}{0pt}
\setlength{\trimedge}{\stockwidth}
\addtolength{\trimedge}{-\paperwidth}
\settypeblocksize{634pt}{448.13pt}{*}
\setulmargins{4cm}{*}{*}
\setlrmargins{*}{*}{1.5}
\setmarginnotes{17pt}{51pt}{\onelineskip}
\setheadfoot{\onelineskip}{2\onelineskip}
\setheaderspaces{*}{2\onelineskip}{*}
\checkandfixthelayout

\usepackage{lmodern}
\usepackage[T1]{fontenc}
\usepackage{pxfonts}
\usepackage{textcomp}

\usepackage[square]{natbib}

\usepackage{graphicx}
\setkeys{Gin}{width=156mm,totalheight=\textheight,keepaspectratio}
\graphicspath{{includes/}}

\title{Sixteen Thoughts on Rule 110}
\author{Jeremy Mates}

\usepackage{hyperref}
\hypersetup{pdfauthor={Jeremy Mates},pdftitle={Sixteen Thoughts on Rule 110}}

\begin{document}
\bibliographystyle{plainnat}

\mainmatter

\begin{center}
\bfseries \Huge \thetitle
\end{center}
\begin{flushright}
\theauthor
\end{flushright}

\includegraphics{39-181a.pdf}%

\includegraphics{57-173a.pdf}%

\includegraphics{78-107a.pdf}%

\includegraphics{114-91a.pdf}%

\includegraphics{147-218a.pdf}%

\includegraphics{156-214a.pdf}%

\includegraphics{201-109a.pdf}%

\includegraphics{228-182a.pdf}%

\includegraphics{39-181b.pdf}%

\includegraphics{57-173b.pdf}%

\includegraphics{78-107b.pdf}%

\includegraphics{114-91b.pdf}%

\includegraphics{147-218b.pdf}%

\includegraphics{156-214b.pdf}%

\includegraphics{201-109b.pdf}%

\includegraphics{228-182b.pdf}%

\chapter*{Code}

\begin{verbatim}
/*
 * Rule 110, 8-bit playfield.
 *
 * NOTE code written on a little-endian system and probably will need
 * porting to a system of some other endianess.
 *
 * Usage:
 *   cc -std=c99 -o gen gen.c && ./gen num
 *
 * where num is a number from 0 to 255 inclusive.
 */

#include <err.h>
#include <errno.h>
#include <limits.h>
#include <math.h>
#include <stdbool.h>
#include <stdint.h>
#include <stdio.h>
#include <stdlib.h>
#include <sysexits.h>
#include <time.h>
#include <unistd.h>

#define FIELD_WIDTH 8
#define OUTCOMES_WIDTH 8

#define IFYOUHAVETOASK 42

int main(int argc, char *argv[])
{
    bool outcomes[OUTCOMES_WIDTH] = { 0, 1, 1, 1, 0, 1, 1, 0 };
    int index;
    uint8_t field1;
    unsigned int iters = 0;
    unsigned int input;

    if (argc != 2)
        errx(EX_USAGE, "Usage: gen {0..255}");

    argv++;                     // skip prog name
    if (sscanf(*argv++, "%3u", &input) != 1)
        errx(EX_USAGE, "Usage: gen {0..255}");
    argc -= 2;
    if (input > UINT8_MAX)
        errx(EX_DATAERR, "input number must be in range 0..%u", UINT8_MAX);
    field1 = input;

    while (++iters < IFYOUHAVETOASK) {
        uint8_t field2 = 0;
        for (unsigned int i = 0; i < FIELD_WIDTH; i++) {
            putchar(((field1 >> i) & 1) ? 'X' : '.');
        }
        printf(" %3u\n", (unsigned int) field1);

        for (int self = 0; self < FIELD_WIDTH; self++) {
            int prev = self - 1;
            int next = self + 1;
            /* Wrap the field around so do not have to worry about
             * (undefined) handling of edge conditions. */
            if (prev < 0)
                prev = FIELD_WIDTH - 1;
            if (next > FIELD_WIDTH - 1)
                next = 0;

            index =
                ((field1 >> prev & 1) << 2) | ((field1 >> self & 1) << 1) |
                (field1 >> next & 1);
            field2 |= outcomes[index] << self;
        }
        field1 = field2;

        if (field1 == 0 || field1 == UINT8_MAX) {
            printf("hilo %u at %u\n", field1, iters);
            exit(0);
        }
    }
    printf("loop %u\n", iters);
    exit(0);
}
\end{verbatim}

\chapter*{Method}

The use of an 8-bit ring buffer keeps output minimal (32-bit has a much larger space to explore), and allows easy graphing of the resulting patterns. The melodies above are based on the eight starting numbers that have a single subsequent step before terminating. There are other interesting patterns in some of the remaining numbers perhaps worth exploring. \\

\begin{verbatim}
$ ./gen 39
XXX..X..  39
X.X.XX.X 181
hilo 255 at 2
\end{verbatim}

Melodization was performed by taking the numbers as binary:

\begin{verbatim}
11100100 39
10101101 181
\end{verbatim}

Then spacing these numbers out:

\begin{verbatim}
 1   1   1   0   0   1   0   0
 1   0   1   0   1   1   0   1
\end{verbatim}

A new number created with the previous, current, and next numbers for each number, plus \texttt{s} and \texttt{e} to indicate the start and end:

\begin{verbatim}
 1   1   1   0   0   1   0   0
s11 111 110 100 001 010 100 00e
 1   0   1   0   1   1   0   1
s10 101 010 101 011 110 101 01e
\end{verbatim}

And then each set (\texttt{s11}, \texttt{111}, etc.) is considered in the construction of a musical phrase. Particular codas\citep[p.29]{taruskin2010} were used for the four \texttt{e} cases, though otherwise the process was fairly loose, with a \texttt{1} usually meaning ascending or faster, and \texttt{0} the opposite, and a variable number of notes or overlap allowed between the different sets. So, a fairly informal process. Whether a cellular automata melody\citep[p.600]{stephenson2008} could be designed that is both musically pleasing and possible to recover the original bits from might make for an interesting study.

\vfill

This work is licensed under the Creative Commons Attribution 3.0 United States License. To view a copy of this license, visit http://creativecommons.org/licenses/by/3.0/us/ or send a letter to Creative Commons, 171 Second Street, Suite 300, San Francisco, California, 94105.

\bibliography{references}

\end{document}
